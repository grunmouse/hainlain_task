
\subsection{Дата старта}
\paragraph{}Наименьшая дата старта равна 12.00 по Гринвичу 15 мая 2087. Она соответствует началу 2483456 юлианского дня.

\subsubsection{Вычисление юлианской даты и аргумента эмпирических функций}
\paragraph{} Для упрощения расчётов, воспользуемся стандартными средствами вычисления Unix-времени, а затем рассчитаем дату по формуле
$$JD = t_{Unix}/86400 + 2440587.5;$$
\paragraph{} 12.00 по Гринвичу 15 мая 2087 соответствует Unix-времени 3703838400 с.
$$JD = 2483456.$$
$$T_u = \frac{2483456 - 2415020}{36525} \approx 1.874.$$
\subparagraph{} Приближенные значения, указанные в тексте, для вычислений не используются, в реальных расчётах промежуточные значения не округляются и не преобразуются в десятичную систему.