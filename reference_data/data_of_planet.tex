\subsection{Параметры орбит планет [1]}
\paragraph{Обозначения}
\subparagraph{}$\Omega$ - эклиптическая долгота восходящего узла,
\subparagraph{}$\omega$ - эклиптический аргумент перигелия,
\subparagraph{}$M$ - средняя аномалия,
\subparagraph{}$L = \Omega + \omega + M$ - средняя долгота,
\subparagraph{}$\widetilde{\omega}$ - долгота перигелия,
\subparagraph{}$e$ - эксцентриситет орбиты, в астрономических единицах,
\subparagraph{}$i$ - наклонение орбиты к плоскости эклиптики,
\subparagraph{}$a$ - большая полуось орбиты,
\subparagraph{}$T$ - момент прохождения перигелия,
\subparagraph{}$n$ - среднее сидерическое движение

\paragraph{Аргумент функций}
$$T_u = \frac{{JD} - {JD}_0}{35625.0};$$
$${JD}_0 - 20415020.0.$$

\stealfile{orbit_venus}
\stealfile{orbit_mars}
\stealfile{orbit_terra}


[1 Эскобал, "Элементы астродинамики", 1971 "Мир"]

