\subsection{Параметры орбит планет[1]}
\paragraph{Обозначения}
\subparagraph{}$\Omega$ - эклиптическая долгота восходящего узла,
\subparagraph{}$\omega$ - эклиптический аргумент перигелия,
\subparagraph{}$M$ - средняя аномалия,
\subparagraph{}$L = \Omega + \omega + M$ - средняя долгота,
\subparagraph{}$\widetilde{\omega}$ - долгота перигелия,
\subparagraph{}$e$ - эксцентриситет орбиты, в астрономических единицах,
\subparagraph{}$i$ - наклонение орбиты к плоскости эклиптики,
\subparagraph{}$a$ - большая полуось орбиты,
\subparagraph{}$T$ - момент прохождения перигелия,
\subparagraph{}$n$ - среднее сидерическое движение

\paragraph{Аргумент функций}
$$T_u = \frac{{JD} - {JD}_0}{35625.0};$$
$${JD}_0 - 20415020.0.$$

\paragraph{Вычисление юлианской даты по Unix-времени}
Для уменьшения количества и сложности используемых формул, мы будем использовать стандартные алгоритмы преобразования дат в Unix-время, а затем находить юлианскую дату по простой формуле:
$$JD = t/86400 + 2440587.5.$$

\paragraph{Параметры орбиты Венеры}
$$L = 342°46'1",39 + 210669162",88 T_u + 1",1148 T_u^2,$$
$$\widetilde{\omega} = 130°9'49",8 + 5098".93 T_u - 3".515 T_u^2,$$
$$\Omega = 75°46'46"73 + 3239".46 T_u + 1".146 T_u^2,$$
$$e = 0.00682069 - 0.00004774 T_u + 0.000000091 T_u^2,$$
$$i = 3°23'37".07 + 3".621 T_u + 0".0035 T_u^2,$$
$$a = 0.72333015.$$

\paragraph{Параметры орбиты Марса}
$$L = 393°44'51" + 68910117".33 T_u + 1".1184 T_u^2,$$
$$\widetilde{\omega} = 334°13'5".53 + 6626".73 T_u - 0".005  T_u^2 - 0.0043  T_u^3,$$
$$\Omega = 48°47'11"16 + 2775".57 T_u - 0".005 T_u^2 - 0".0192 T_u^3,$$
$$e = 0.09331290 + 0.000092064 T_u - 0.000000077 T_u^2,$$
$$i = 1°51'1".20 - 2".430 T_u + 0".0454 T_u^2,$$
$$a = 1.52368839.$$

[1 Эскобал, "Элементы астродинамики", 1971 "Мир"]

\subsection{Параметры орбит отлёта и назначения}
\paragraph{Орбита отлёта}
За орбиту отлёта примем круговую марсоцентрическую орбиту, радиусом 23458 км, в расположенную в плоскости, наиболее удобной для отлёта - полагаем, что плоскость орбиты будет сменена заблаговременно.

\paragraph{Орбита назначения}
У Венеры нет стратосферы, поэтому предположим, что орбита назначения располагается на высоте порядка 500 км - с некоторым запасом выше плотных слоёв атмосферы. Что соответствует радиусу орбиты порядка 6550 км. Плоскость орбиты также будем полагать наиболее удобной для прилёта.

\subsection{Параметры поля тяготения планет и Солнца}
\paragraph{Константы и формулы}
$$\gamma = 6.67408e-11 \pm 0.00031e-11 [2]$$
%/* м^3/(кг*с^2) */

$$\mu = \gamma M$$
$$v_{par} = \sqrt{\frac{2\gamma M}{R}}$$
\subparagraph{}По соображениям точности вычислений, промежуточные расчёты в десятичную систему счисления переводиться не будут, и в статье не будут представлены.

\paragraph{Массы планет [3]}
\subparagraph{Солнце} $M = 1.9885e+30 kg$
\subparagraph{Венера} $M = 4,8675e+24 kg$
\subparagraph{Марс} $M = 6,4171e+23 kg$

[3 Википедия, со ссылкой на НАСА]
[2 CODATA2014 https://physics.nist.gov/cgi-bin/cuu/Value?bg]