\subsection{Порядок величины времени полёта}

\paragraph{}Полуоси орбит планет в а.е.:
$$a_{Venus} = 0.72333015$$
$$a_{Mars} = 1.52368839$$

\paragraph{}Прикинем Гомановскую орбиту:
$$a = \frac{a_{Venus} + a_{Mars}}{2} \approx 1.1235.$$
\subparagraph{}Т.к. большая полуось у нас в а.е., период обращения в годах мы можем получить по простой формуле:
$$T = a^{3/2} \approx 1.190$$
\subparagraph{}Время полёта - полупериод, соответственно:
$$\tau  \approx 0.595$$
Что примерно равно 217 дням, и соответствует приращению аргумента
$$\Delta T_u \approx 0.0061$$

\paragraph{}Прикинем изменение параметров орбиты Венеры за это время
\subparagraph{}Т.к. $\Delta T_u$ очень мало, можно пренебречь членами ряда, содержащим его квадрат и куб.\
\subparagraph{}Эксцентриситет за время полёта изменится на величину $0.000092064 T_u$ или 5.6e-7, что соответствует изменению полуфокусного расстояния на 60 км, этим можно пренебречь.
\subparagraph{}Из угловых параметров орбиты, наиболее быстро изменяется $\widetilde{\omega}$ - $5098".93 T_u$, что за время полёта составит $31.1"$ или 0.000150 радиана. Это соответствует смещению перигелия на 0.000109 а.е. или 16 тыс.км.

\subparagraph{}Иными словами, можно считать, что за время полёта, параметры орбиты Венеры почти не изменяются, и отдельный их расчёт для конца пути целесообразен только с точки зрения уточнения параметров коррекции траектории.