\section{Постановка задачи}

\subsection{Цитата из книги}
\subparagraph{}Задача гласила: «Дано: вылет с орбиты Деймоса, спутника Марса, не раньше 12.00 по Гринвичу 15 мая 2087; двигатели космического корабля работают на химическом топливе, скорость выброса газов 10 тысяч метров в секунду: место назначения — супрастратосферная орбита вокруг Венеры. Требуется рассчитать: наиболее экономичный курс полета к месту назначения при наименьшей затрате времени, соотношения массы, а также время вылета и прибытия. Подготовить полетный план и вычислить точки коррекции, используя звезды второй величины или более яркие. Вопросы: можно ли сберечь время и топливо, пролетев рядом с парой планет Терра-Луна, и получить от их гравитационного поля дополнительное ускорение? Предвидится ли пересечение метеорных потоков и какие меры по уклонению от столкновения с метеорами будут предприняты? Все ответы должны соответствовать правилам космических полетов, а также принципам космической баллистики».
\subparagraph{}Задачу нельзя было решить за короткий отрезок времени без помощи компьютера. Тем не менее, Мэтт мог подготовить соответствующие уравнения и затем, если повезет, уговорить офицера, руководящего вычислительным центром Лунной базы, разрешить ему воспользоваться баллистическим интегратором.

\subsection{Разбор}
\paragraph{}У Венеры нет стратосферы, поэтому положим, что имеется в виду низкая орбита.
\paragraph{}Космос - не море, здесь нет постоянного курса. Будем считать, что имеются в виду Кеплеровы параметры пассивных участков траектории и запланированные на ней манёвры.
\paragraph{}Наиболее экономичный курс и наименьшее время полёта - взаимоисключающие параграфы. Сделаем несколько решений, оптимизированных по разным параметрам, причём, время полёта будем считать вместе с временем ожидания окна, начиная с указанной наименьшей даты старта.
\paragraph{}Никаких правил космических полётов автор не приводит, поэтому просто опустим этот пункт.
\paragraph{}Время вылета не ограничено сверху.
\paragraph{}По остальному - всё нормально.

\subsection{Формализация}
\subsubsection{Дано:}
\paragraph{Удельный импульс:} 10000 м/с.
\paragraph{Наименьшее время старта:} 12.00 по Гринвичу 15 мая 2087.
\paragraph{Орбита отлёта:} Примем круговую марсоцентрическую орбиту, радиусом 23458 км, в расположенную в плоскости, наиболее удобной для отлёта - полагаем, что плоскость орбиты будет сменена заблаговременно.
\paragraph{Орбита назначения:} Предположим, что орбита назначения располагается на высоте порядка 500 км - с некоторым запасом выше плотных слоёв атмосферы. Что соответствует радиусу орбиты порядка 6550 км. Плоскость орбиты также будем полагать наиболее удобной для прилёта.

\subsubsection{Надо:}
\paragraph{1.} Найти параметры межпланетной траектории, наиболее экономичной по топливу, рассчитать соотношения массы для неё.
\paragraph{2.} Найти параметры наиболее быстрой межпланетной траектории.